\documentclass[a4paper,14pt]{article}

% Поддержка русского языка
\usepackage[T1, T2A]{fontenc}    % Кодировка
\usepackage[utf8]{inputenc}  % Кодировка исходного текста
\usepackage[english, russian]{babel}  % Локализация и переносы
\usepackage{titlesec, titletoc} % Для настройки стилей заголовков и содержания
\usepackage{booktabs} % Для использования \toprule, \midrule и \bottomrule
\usepackage{array}  % Для лучшего контроля столбцов
\usepackage{cmap}    % Поиск и копирование в PDF
\usepackage{geometry}    % Поля
    \geometry{left=30mm, right=15mm, top=20mm, bottom=20mm}
\usepackage{hhline} % Для двойных линий и более точного контроля границ
\usepackage{caption} % Для настройки подписей
\usepackage{verbatim}
\usepackage{xcolor} % Пакет для цвета
\usepackage{listings} % Пакет для вставки кода
\usepackage{indentfirst}
\usepackage{graphicx}
\usepackage{float}
\usepackage{amsmath}
\usepackage{tikz}

\usetikzlibrary{shapes.geometric, arrows}

\tikzstyle{startstop} = [rectangle, rounded corners, minimum width=3.5cm, minimum height=1cm, text centered, draw=black, fill=red!30]
\tikzstyle{process} = [rectangle, minimum width=3.5cm, minimum height=1cm, text centered, draw=black, fill=blue!20]
\tikzstyle{decision} = [diamond, minimum width=3.5cm, minimum height=1cm, text centered, draw=black, fill=green!30]
\tikzstyle{arrow} = [thick,->,>=stealth]

% \usepackage{times}

% Установка стандартного отступа (2em)
\setlength{\parindent}{2em}

% Установка отступа после заголовков
\usepackage{titlesec}
\titlespacing*{\section}{0pt}{2\baselineskip}{\baselineskip}
\titlespacing*{\subsection}{0pt}{2\baselineskip}{\baselineskip}
\titlespacing*{\subsubsection}{0pt}{2\baselineskip}{\baselineskip}

% Определение цветов
% \definecolor{operatorcolor}{HTML}{ED028C}
% \definecolor{stringcolor}{HTML}{9400D1}
% \definecolor{commentcolor}{HTML}{005000}
% \definecolor{backgroundcolor}{HTML}{fafafa}
\definecolor{backgroundcolor}{HTML}{fafafa} % Темный фон
\definecolor{keywordcolor}{rgb}{0.86, 0.58, 0.35} % Ключевые слова (оранжевый)
\definecolor{stringcolor}{rgb}{0.72, 0.92, 0.53} % Строки (зеленый)
\definecolor{commentcolor}{HTML}{005000} % Комментарии (серый)
\definecolor{numbercolor}{rgb}{0.75, 0.75, 0.75} % Номера строк (светло-серый)
\definecolor{identifiercolor}{rgb}{0.60, 0.60, 1.00} % Идентификаторы (светло-синий)
\definecolor{keywordtypcolor}{rgb}{0.57, 0.81, 1.00} % Типы данных (голубой)

\captionsetup[table]{justification=raggedright, singlelinecheck=false} % Выравнивание подписи по левому краю

% Установка стиля заголовков разделов: центрирование без номера
\titleformat{\section}[block]{\Large\bfseries\centering}{}{0pt}{}

\title{3.1 Титульный лист}

\begin{document}

\thispagestyle{empty}    % Отключаем колонтитулы

\begin{center}
    ФЕДЕРАЛЬНОЕ ГОСУДАРСТВЕННОЕ АВТОНОМНОЕ ОБРАЗОВАТЕЛЬНОЕ УЧРЕЖДЕНИЕ ВЫСШЕГО ОБРАЗОВАНИЯ\\
    \bfseries{САНКТ-ПЕТЕРБУРГСКИЙ ПОЛИТЕХНИЧЕСКИЙ УНИВЕРСИТЕТ ПЕТРА ВЕЛИКОГО}\\
    Институт компьютерных наук и кибербезопасности\\
    Высшая школа программной инженерии
\end{center}

\vspace{20ex} % Задаем размер вертикального промежутка в явном виде

\begin{center}
    \begin{huge} {\bfseries{\scshape лабораторная работа №2}} \end{huge}

    \vspace{3ex}
    по дисциплине: «Вычислительная матиматика»\\
    Вариант №27
\end{center}

\vspace{30ex}

\noindent Выполнила\\
студентка гр. в5130904/30022\hfill \begin{minipage}{0.6\textwidth} \hfill Г.М.Феллер\end{minipage}

\vspace{3ex}

\noindent Преподаватель \hfill \begin{minipage} {0.6\textwidth}\hfill С.П.Воскобойников\end{minipage}

\vspace{3ex}

\hfill \begin{minipage}{0.6\textwidth} \hfill «\underline{\hspace{1cm}}»\underline{\hspace{3cm}} 2025 г.\end{minipage}

\vfill

\begin{center}
    Санкт-Петербург\\ 
    2025
\end{center}

\newpage % Начинаем новую страницу\

% Отключение нумерации разделов в заголовках разделов и подразделов
\titleformat{\section}[block]{\Large\bfseries\centering}{}{0pt}{}

% Настройка заголовков подразделов с использованием символа "§" и собственной нумерации
\renewcommand{\thesubsection}{\S\arabic{subsection}}
\titleformat{\subsection}[block]{\large\bfseries}{\thesubsection}{1em}{}
% Установка автоматического сброса счетчика подразделов при начале нового раздела

% Определение стиля оглавления
\titlecontents{section}[0pt] % Left indent
{\vspace{0.5ex}\hspace{1em}} % Above code and left spacing
{} % Numbered entry format (empty to remove numbers)
{} % Numberless entry format
{\titlerule*[1pc]{.}\contentspage} % Filler-page format

\newpage

\section{Задание}

Написать процедуру формирования матрицы $A$ по заданному вектору $B$
$$
pA = \begin{pmatrix}
        1     & a_1   & a_1   & \dots & a_1     \\
        1     & 1     & a_2   & \dots & a_2     \\
        \dots & \dots & \dots & \dots & \dots   \\
        1     & 1     & 1     & \dots & a_{n-1} \\
        1     & 1     & 1     & \dots & 1
    \end{pmatrix},
B = \begin{pmatrix}
        a_1 & a_2 & \dots & a_{n-1} \\
    \end{pmatrix}^T
$$

Задавая $n = 5, a_1 = 4, a_2 = 3, a_3 = 2, a_4 = var = 1.5; 1.01; 1.001; 1.0001$ и вычисляя $A^{-1}$ с помощью DECOMP и SOLVE, найти нормы матриц $R = AA^{-1} - E$ для всех вариантов $a_4$.



\end{document}